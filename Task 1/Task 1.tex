\documentclass[12pt]{report}
\usepackage{pgf}
\usepackage{pgfpages}
\usepackage{graphicx}
\usepackage{geometry}
 \geometry{
 a4paper,
 total={170mm,257mm},
 left=20mm,
 top=20mm,
 }

\pgfpagesdeclarelayout{boxed}
{
  \edef\pgfpageoptionborder{0pt}
}
{
  \pgfpagesphysicalpageoptions
  {%
    logical pages=1,%
  }
  \pgfpageslogicalpageoptions{1}
  {
    border code=\pgfsetlinewidth{2pt}\pgfstroke,%
    border shrink=\pgfpageoptionborder,%
    resized width=.95\pgfphysicalwidth,%
    resized height=.95\pgfphysicalheight,%
    center=\pgfpoint{.5\pgfphysicalwidth}{.5\pgfphysicalheight}%
  }%
}

\pgfpagesuselayout{boxed}


\begin{document}

\begin{titlepage}
   \vspace*{\stretch{1.0}}
   \begin{center}
      \Large\textbf{Assignment No. 1}\\
      \large\textit{Archan Mehta (201711014)}
      \large\textit{\\*Dharit Parikh (201711022)}
   \end{center}
   \vspace*{\stretch{2.0}}
\end{titlepage}

\begin{center}
\begin{enumerate}

\item Write a MATLAB function myresize.m which can scale an input image to any given size (M,N). Use bilinear interpolation for this purpose. Show couple of results in your report

\begin{figure}[h]
\includegraphics[width=\textwidth]{assign1out.JPG}
\caption{300x250 image resize}
\end{figure}

\begin{figure}[h]
\includegraphics[width=\textwidth]{assign1out2.jpg}
\caption{900x850 image resize}
\end{figure}

\pagebreak

\item Write a MATLAB function myrotate.m which can roate an input image at any user specified angle about the center of the image. Show a couple of results in your report and verify whether rotating any image n times by an angle \[\theta = \frac{2\pi}{n}\]
for some fixed n, yields the original image or not.

\begin{figure}[h]
\includegraphics[width=\textwidth]{assign2.jpg}
\caption{rotated image}
\end{figure}


\end{enumerate}
\end{center}

\end{document}
