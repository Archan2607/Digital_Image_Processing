\documentclass[12pt]{report}
\usepackage{pgf}
\usepackage{pgfpages}
\usepackage{graphicx}
\usepackage{geometry}
 \geometry{
 a4paper,
 total={170mm,257mm},
 left=20mm,
 top=20mm,
 }

\pgfpagesdeclarelayout{boxed}
{
  \edef\pgfpageoptionborder{0pt}
}
{
  \pgfpagesphysicalpageoptions
  {%
    logical pages=1,%
  }
  \pgfpageslogicalpageoptions{1}
  {
    border code=\pgfsetlinewidth{2pt}\pgfstroke,%
    border shrink=\pgfpageoptionborder,%
    resized width=.95\pgfphysicalwidth,%
    resized height=.95\pgfphysicalheight,%
    center=\pgfpoint{.5\pgfphysicalwidth}{.5\pgfphysicalheight}%
  }%
}

\pgfpagesuselayout{boxed}


\begin{document}

\begin{titlepage}
   \vspace*{\stretch{1.0}}
   \begin{center}
      \Large\textbf{Assignment No. 2}\\
      \large\textit{Archan Mehta (201711014)}
      \large\textit{\\*Dharit Parikh (201711022)}
   \end{center}
   \vspace*{\stretch{2.0}}
\end{titlepage}

\begin{center}
\begin{enumerate}

\item Implement the Histogram specification algorithm using the two approaches discussed
in class: (a) via T−12· T1, and (b) exact histogram specification. Name the two files
myhistspec.m and myexacthistspec.m. In the first case, the input to the function
should be two vectors pR and pS, the input and specified distribution functions, each of
size L × 1, while the output should be a vector containing T−12· T1(rk), k = 0, . . . , L − 1.
In the second case, the input to the function should be an image I, the specified distribution
function pS as an L × 1 vector, and k, the number of neighborhoods to be
considered for inducing a total order on the pixels. The neighborhoods used should
be of size (2j + 1) × (2j + 1), j = 1, . . . , k. The output of this function should be the
output image with the desired distribution pS, only if the given k number of neighborhoods
induce a total order. Otherwise, the function should simply give out the error
message: The neighborhoods do not induce a total order on the pixels of
the given image.

\begin{figure}[h]
\includegraphics[width=\textwidth]{1_a.JPG}
\caption{Original Image 1}
\end{figure}

\begin{figure}[h]
\includegraphics[width=\textwidth]{1_b.jpg}
\caption{Histogram of Original Image 1}
\end{figure}


\begin{figure}[h]
\includegraphics[width=\textwidth]{1_c.jpg}
\caption{Equalized Image 1}
\end{figure}



\begin{figure}[h]
\includegraphics[width=\textwidth]{1_d.jpg}
\caption{Histogram Of Equalizaed Image 1}
\end{figure}


\begin{figure}[h]
\includegraphics[width=\textwidth]{1_a_1.JPG}
\caption{Original Image 2}
\end{figure}

\begin{figure}[h]
\includegraphics[width=\textwidth]{1_b_1.jpg}
\caption{Histogram of Original Image 2}
\end{figure}


\begin{figure}[h]
\includegraphics[width=\textwidth]{1_c_1.jpg}
\caption{Equalized Image 2}
\end{figure}



\begin{figure}[h]
\includegraphics[width=\textwidth]{1_d_1.jpg}
\caption{Histogram Of Equalizaed Image 2}
\end{figure}


\pagebreak



\end{enumerate}
\end{center}

\end{document}
